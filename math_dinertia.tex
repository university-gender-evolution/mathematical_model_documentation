
\documentclass[12pt, a4paper, oneside, headinclude, footinclude,english]{article}

\input{/home/krishnab/Dropbox/backup/templates/latex/math_notes/structure.tex}
\input{/home/krishnab/Dropbox/backup/templates/latex/math_macros/math_macros.tex}
\title{\normalfont\spacedallcaps{Mathematical Model for Demographic Inertia in
    Faculty Hiring}}

\author{\spacedlowsmallcaps{Krishna Bhogaonker}} 
 
\date{\today\\version 0.1}
\makeatletter
%\providecommand{\@nil}
\makeatother


\begin{document}

\renewcommand{\sectionmark}[1]{\markright{\spacedlowsmallcaps{#1}}} 
\lehead{\mbox{\llap{\small\thepage\kern1em\color{halfgray} \vline}\color{halfgray}\hspace{0.5em}\rightmark\hfil}} 



\pagestyle{scrheadings} 

%----------------------------------------------------------------------------------------
%	TABLE OF CONTENTS & LISTS OF FIGURES AND TABLES
%----------------------------------------------------------------------------------------

\maketitle 

\setcounter{tocdepth}{3}

\tableofcontents 

% \listoffigures 

% \listoftables

%----------------------------------------------------------------------------------------
%	ABSTRACT
%----------------------------------------------------------------------------------------

\section*{Abstract}

The past few decades have witnessed substantial improvements in the gender
diversity of newly minted Ph.Ds from top universities. At the same time, the
gender diversity of the faculty in these same universities has not changed to
reflect this change in candidate supply. The model developed below examines the
change in faculty gender diversity over time, given as set of initial conditions
and model parameters. 
%----------------------------------------------------------------------------------------
%	AUTHOR AFFILIATIONS
%----------------------------------------------------------------------------------------

%\let\thefootnote\relax\footnotetext{* \textit{}}

%\let\thefootnote\relax\footnotetext{\textsuperscript{1} \textit{}}

%----------------------------------------------------------------------------------------

% \newpage 

%----------------------------------------------------------------------------------------
%	INTRODUCTION
%----------------------------------------------------------------------------------------

\section{Notation}
\label{sec:notation}

This is a list of some of the mathematical notation used in the equations below. 

\begin{table}[H]
\centering
\begin{tabular}{|c|l|c|l|}
\hline
 \multicolumn{1}{|c|}{\textbf{Variable}} & \multicolumn{1}{|c|}{\textbf{Description}} & \multicolumn{1}{|c|}{\textbf{Range}} \\ \hline
 $R$ & the set of professor ranks: Assistant, Associate, Full & $\{1,2,3\}$ \\ \hline
 $r$ & rank of a particular group of professors & $r \in R$   \\ \hline
 $N$ & total number of professors in the department & $N \in \zp$   \\ \hline
  $n^r$ & number of professors at a given rank & $n^r \in \zp$   \\ \hline
  $G$ & the set of professor genders & $\{m, f\}$ \\ \hline
  $g$ & the gender of a particular professor & $g \in G$ \\ \hline
  $q^r$ & target percentage of professors at rank $r$ & $q^r \in [0,1]$  \\ \hline
  $T^N$ & target number of the size of the department & $T^N \in \zp$  \\ \hline
 $v^{r}$ & number of vacancies at rank $r$ & $v^r \in \zp$  \\ \hline
 $v^N$ & total number of vacancies in the department & $v^N \in \zp$  \\ \hline
 $h^{r}_{g}$ & number of professors hired at rank $r$ and gender $g$ & $h^{r}_{g} \in \zp$ \\ \hline
 $s$ & yearly growth rate for the department & $\{s \in R | -1 \leq s \leq 1\}$ \\ \hline
\end{tabular}
\end{table}


\section{Gender Diversity Evolution Model}
\label{sec:genduniv-model}

In the next two sections we lay out the structure and dynamics of the
demographic inertia model for gender balance in academic faculty. In the first
section we describe the organization and actors in an academic department. Based
upon this organization we outline the evaluation metric to study the
department's gender balance. In the second section we define the dynamics of the
model, namely the way that faculty join and enter the department on a year by
year basis. As we simulate faculty turnover, we evaluate the evolution of gender
balance using our evaluation metric. 

\subsection{Academic Department Composition and Structure}
\label{sec:dept-structure}

An academic department is composed of both male and female faculty that teach
and conduct research in a specific discipline. Each professor has a rank
contingent on his/her position within the department's status ladder. The most junior rank
is the Assistant professor who is a tenure-track candidate, but who has not yet
achieved tenure. In most cases tenure takes about 6 years from the initial hire
date of the professor. The second rank is the Associate professor who has
achieved tenure, but who lacks the publication/teaching/committee
record to achieve the level of full professor. The final rank is Full professor,
which is the apex of the promotion ladder.

To formalize the description above, we define $\bm{n^r}$ as the number of professors
at a given rank, where rank $1$ represents an Assistant professor and rank $3$ represents
a Full professor. The number $\bm{n^r}$ is equal to the sum of male $\bm{n_m^r}$ and
female $\bm{n_f^r}$ professors at that rank, in the department. At any given rank,
the number of professors is equal to:
\begin{align}
  n^1 & = n^1_f + n^1_m \label{eq:dept-size-1} \\
  n^2 & = n^2_f + n^2_m  \label{eq:dept-size-2} \\
  n^3 & = n^3_f + n^3_m \label{eq:dept-size-3}
\end{align}

The total size of the department ($\bm{N}$) is equal to the number of professors at
each rank:

\begin{equation}
  \label{eq:total-dept-size}
  N = n^1 + n^2 + n^3
\end{equation}

By substituting equations
\ref{eq:dept-size-1}, \ref{eq:dept-size-2}, and \ref{eq:dept-size-3} into equation \ref{eq:total-dept-size} we obtain a
gendered version of department size equation where the number of males in the
department is $\bm{n_m}$, and the number of females in the department is $\bm{n_f}$:

\begin{align}
  n_m &= n^1_m + n^2_m + n^3_m  \label{eq:female-number} \\
  n_f &= n^1_f + n^2_f + n^3_f  \label{eq:male-number} \\
  N &= n_f + n_m  \label{eq:total-gender-dept-size}
\end{align}

Our intent in this project is to study the evolution of department gender balance
over time, meaning the balance of male and female faculty. To measure this
balance we use a elementary metric: The percentage of female
faculty, as derived from equations \ref{eq:female-number} and \ref{eq:total-gender-dept-size}. 

\begin{align}
  \label{eq:pct-female}
  \text{Percentage of female faculty} &= \frac{n_f}{N} \label{eq:pct-female} \\
  \text{Percentage of male faculty} &= 1 - \frac{n_f}{N}          
\end{align}

As the department undergoes yearly faculty changes and turnover--as described in
the section below--we track how these simulated changes will affect the
percentage of female faculty in the department. This percentage also informs our
analysis as to the potential for achieving reasonable gender equity--in faculty
composition--given current university gender diversity enhancement policies.

\subsection{Department Dynamics}
\label{sec:dept-dynamics}

Of course the department's membership and faculty composition does not remain
constant over time. Every year faculty leave due to retirement or
job offers from other universities. New faculty join the department through a
hiring process. A more appropriate way to write the equations above is to
include a reference to a given timestep. In other words, we could write equation
\ref{eq:total-dept-size} with an explicit reference to time as below:

\begin{equation*}
  N(t) = n^1(t) + n^2(t) + n^3(t) \label{eq:test}
\end{equation*}

The discussion on time naturally leads to an explanation for the dynamics of the
model. We envision three steady-state processes of entry and exit. The three processes include:
attrition, hiring, and promotion. The equations are a recursive definition of
the number of male and female professors at each rank at time $t+1$, based upon
changes occurring at time $t$. The equations below define a system of discrete
linear difference equations. 


\begin{align} 
  n^1_f(t+1) &= n^1_f(t) - a^1_f(t) + h^1_f(t) - b^{1,2}_f(t)  \label{eq:nf1} \\
  n^1_m(t+1) &= n^1_m(t) - a^1_m(t) + h^1_m(t) - b^{1,2}_m(t) \label{eq:nm1} \\
  n^2_f(t+1) &= n^2_f(t) - a^2_f(t) + h^2_f(t) - b^{2,3}_f(t) + p^{1,2}_f(t) \label{eq:nf2} \\ 
  n^2_m(t+1) &= n^2_m(t) - a^2_m(t) + h^2_m(t) - b^{2,3}_m(t) + p^{1,2}_m(t)  \label{eq:nm2} \\ 
  n^3_f(t+1) &= n^3_f(t) - a^3_f(t) + h^3_f(t) + b^{2,3}_f(t) \label{eq:nf3} \\
  n^3_m(t+1) &= n^3_m(t) - a^3_m(t) + h^3_m(t) + b^{2,3}_m(t) \label{eq:nm3}
\end{align}

In equations \ref{eq:nf1}-\ref{eq:nm3} above, the $\bm{a^r_g}$ term refer to the number of professor
attritions in year $t$, while the $\bm{h^r_g}$ term refers to the number of
professor hires. The $\bm{b^{r, r+1}}$ terms refer to promotions between
professor levels. As per the equations, the number of professors promoted from a
given rank will augment the faculty numbers of the higher rank and decrement the
faculty numbers of their current rank. Further, professors cannot be promoted
into the Assistant professor rank and cannot be promoted beyond the Full
professor rank for obvious reasons. 

Next we refine equations \ref{eq:nf1}-\ref{eq:nm3} by modeling the calculation of
attrition, hiring, and promotions. In modeling each of these processes, we also
explain how factors like department growth targets and exogenous faculty churn
also affect department faculty numbers and gender balance. 

\subsubsection{Attrition Process($a^r$)}
\label{sec:attrition}

As professors retire from the department or leave due to new opportunities, the
department size will fall. These departures occur at all ranks, though most
attrition occurs at the full professor level--since this group represents the
largest group of professors in the department. The downward increment in
department size at a given rank and gender is denoted $\bm{a^r_g}$. The number
of attritions $a^r_g \in \zp$ at a given rank and gender are defined by the
following system of equations.

\begin{align}
  a^1_f(t) &= \binomial{n^1_f(t)}{p^{1, attr}_f} + \binomial{cn^1_f(t)}{\frac{n^1_f(t)}{N(t)}}  \label{eq:attr_f1}\\
  a^1_m(t) &= \binomial{n^1_m(t)}{p^{1, attr}_m} + \binomial{cn^1_m(t)}{\frac{n^1_m(t)}{N(t)}} \label{eq:attr_m1} \\
  a^2_f(t) &= \binomial{n^2_f(t)}{p^{2, attr}_f} + \binomial{cn^2_f(t)}{\frac{n^2_f(t)}{N(t)}} \label{eq:attr_f2} \\
  a^2_m(t) &= \binomial{n^2_m(t)}{p^{2, attr}_m} + \binomial{cn^2_m(t)}{\frac{n^2_m(t)}{N(t)}} \label{eq:attr_m2} \\
  a^3_f(t) &= \binomial{n^3_f(t)}{p^{3, attr}_f} + \binomial{cn^3_f(t)}{\frac{n^3_f(t)}{N(t)}} \label{eq:attr_f3} \\
  a^3_m(t) &= \binomial{n^3_m(t)}{p^{3, attr}_m} + \binomial{cn^3_m(t)}{\frac{n^3_m(t)}{N(t)}} \label{eq:attr_m3} 
\end{align}

The first time in equation is the structural number of attritions. Since
attritions must be integers, we use a Binomial distribution which determines the
number of successes in a given number of trials. The benefit of using a
probability distribution is that it adds a stochastic character to the model
outcomes. We use the empirically determined attrition rate for each rank and
group $\bm{p^{r,attr}_g}$, in each binomial distribution. Notably, the attrition
rates do not sum to one, otherwise there would an unrealistic level of attrition
each year.

The second term represents random churn or turnover in the department. In some
years, there may be more or less attritions than expected. This term modulates
additional stochasticity to evaluate how perturbations affect the stability of
the system's outputs. The level of churn is determined with regard to the
overall department size. Hence the adjustable parameter $\bm{c}$ represents the
fraction of the department that is susceptible to random churn, for example 5 or 10
percent. $N$ represents the size of the entire department and the distribution
of the additional stochasticity is proportional to the fraction of the
department size that each group constitutes. 

\subsubsection{Hiring Process ($h^r_g$)}
\label{sec:hiring}

The hiring process also must produce an integer number of hires, and hence
relies on a binomial process as well.


\begin{align}
  h^1_f(t) &= \binomial{A(t) + cn^1_f(t) + gN(t)}{p^{1, hiring}_f}  \label{eq:hiring_f1}\\
  h^1_m(t) &= \binomial{A(t) + cn^1_m(t) + gN(t)}{p^{1, hiring}_m}  \label{eq:hiring_m1}\\
  h^2_f(t) &= \binomial{A(t) + cn^2_f(t) + gN(t)}{p^{2, hiring}_f}  \label{eq:hiring_f2}\\
  h^2_m(t) &= \binomial{A(t) + cn^2_m(t) + gN(t)}{p^{2, hiring}_m}  \label{eq:hiring_m2}\\
  h^3_f(t) &= \binomial{A(t) + cn^3_f(t) + gN(t)}{p^{3, hiring}_f}  \label{eq:hiring_f3}\\
  h^3_m(t) &= \binomial{A(t) + cn^3_m(t) + gN(t)}{p^{3, hiring}_m}  \label{eq:hiring_m3}\\
  &\sum_{r \in R}\sum_{g \in G}p_g^{r, hiring} = 1 \label{eq:hiring_constraint}
\end{align}

The first term in the the equations \ref{eq:hiring_f1}-\ref{eq:hiring_m3} is the
total number of attritions in the department $A$. The first task of the hiring
process is to fill any voids left by the departures of professors. The second
term in the hiring equation is again a churn term, to model any additional
volatility or stochasticity in the number of possible hires in a given year. The
final term is a growth term $gN(t)$, as departments tend to grow or shrink in
size over time, given prevailing university investments and budgeting
priorities. The $g$ in the equation is the growth rate for the department. The
growth rate times the department size indicates the number of additional
professors to hire in excess of any attritions that need filling. This term
$gN(t)$ must be rounded to the nearest integer value.

The hiring rates $p^{r, hiring}_g$ reflect the tendency for the department to
hire most professors at the Assistant professor level, though the majority of
attritions occur at the Full professor level. This aspect of the model mirrors
the actual hiring processes in most academic departments.

One aspects of the hiring equations bear some explanation. In each equation the
total attrition term is the same, namely $A$. While this may seem like the
equation hires $A$ number of individuals at each level, this is not the case.
Because the hiring rates are different by group, and these rates sum to $1$, the
total number of hires is equal to $A$--independent of any contribution from the
churn and growth terms. These hires are simply peppered over the levels
commensurate with the hiring rates for each group and level. The hires
associated with churn and growth are similarly distributed over the department
groups. 

\subsubsection{Promotion Process($b^r_g$)}
\label{sec:promotion}

The promotion process simply redistributes professors between levels instead of
adding or removing professors from the department. The promotion process simply
promotes a certain fraction of professors at the Assistant or Associate levels
to the their respective increments in each year.

\begin{align}
  b^1_f(t) &= \binomial{n^1_f(t)}{p^{1,2}_f}  \label{eq:attr_f1}\\
  b^1_m(t) &= \binomial{n^1_m(t)}{p^{1, 2}_m}  \label{eq:attr_m1} \\
  b^2_f(t) &= \binomial{n^2_f(t)}{p^{2, 3}_f}  \label{eq:attr_f2} \\
  b^2_m(t) &= \binomial{n^2_m(t)}{p^{2, 3}_m}  \label{eq:attr_m2}
\end{align}


\section{Differences Between New Model and Previous Model}
\label{sec:differences}

The formulation of the current model resolves some issues and inconsistencies in
the previous model. In this section we identify the differences between the
current models and the previous model. The governing system of linear difference
equations \ref{eq:nf1}-\ref{eq:nm3} on page \pageref{eq:nf1} have not changed
between models. The same processes of attrition, hiring, and promotion
contribute to the dynamics of the model as before.

These differences between models cluster around the way that the churn and
growth processes are included in the model. In the original model, the a single
churn variable captured both hiring and attrition. That variable took states of
$-1, 0, 1$, where $-1$ roughly represented one extra attrition and $+1$
represented roughly one additional hire. However this formulation led to an unintended
coupling between the attrition and hiring processes, a coupling that is not
reflective of the actual processes in an academic department. The new model
introduces a separate churn term for the attrition process and a separate churn
term for the hiring process. Both churn terms follow the same distribution, but
in opposite directions. Hence as long as the sum of both terms averages to zero,
the churn processes add no bias or drift to the model.

The second change is the decoupling of the growth process from the churn
process. In the previous model the number of additional FTEs due to growth was
added to the number of FTE changes due to the churn process. This had the
unintended consequence of coupling growth to both the attrition and hiring
processes as well. While a correlation between hiring and growth is
unavoidable--since both processes add professors to the department, the
correlation with attrition was unintended. The new model removes this coupling. 



%----------------------------------------------------------------------------------------
%	BIBLIOGRAPHY
%----------------------------------------------------------------------------------------

\renewcommand{\refname}{\spacedlowsmallcaps{References}} 

\bibliographystyle{unsrt}

\bibliography{/home/krishnab/Dropbox/bibtex/library.bib}

%----------------------------------------------------------------------------------------
\end{document}
